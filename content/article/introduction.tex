\section{Introduction}
\label{intro}
Skin cancer is the most prevalent form of human malignancy, with an increasing incidence rate over the years. This pathology can have a heavy social impact on those affected, not only decreasing their quality of life, but also potentially becoming lethal. In addition, it has significant economic consequences, with an estimated cost of 8 billion dollars per year in the United States~\cite{Farberg2017a}. However, most of these repercussions could be avoided with an early detection and appropriate surgeries~\cite{Farberg2017a}.\par
Until these days, biopsies are the gold standard to check skin pathology. However, they still remain time consuming, invasive and inconvenient for experts and patients. Consequently, several imaging techniques were developed to perform an early detection of these diseases, some of which are of common use by experts. For instance, Clinical Photography and Dermatoscopy are both examples of affordable, easy to use techniques largely exploited by dermatologist. Dermatoscopy tends to replace Clinical Photography as it significantly improves the quality of diagnosis made by experts, thanks to acquisition of homogeneous images~\cite{Sinz2017}.\par
Many research papers based on automatic classification for dermatology focus on dermatoscopy. Most of them obtain acceptable results on Melanocytic Skin Cancer pathologies~\cite{Iyatomi2010}. Older methods focused on finding the most pertinent combination of preprocessed and hand-crafted features, to be used in a machine learning scheme~\cite{Rastgoo2015}~\cite{Pathan2018}. In contrast, recent methods based on Deep Learning techniques appear to show impressive results~\cite{Esteva2017}.\par
\ac{rcm} is another type of imaging technique used by dermatologists and is more efficient for both diagnosis of Melanocytic and Non Melanocytic lesions~\cite{Gerger2006}~\cite{Guitera2009}~\cite{Haroon2017}. Furthermore, this modality can provide slices at different depths of the skin by adjusting wavelength property and focal point~\cite{Kolm2012}. In opposition to previous modalities, \ac{rcm} remains expensive, although the number of users continue to increase~\cite{Batta2015}. In recent years, researchers started to improve the portability of \ac{rcm} devices~\cite{Freeman2018}.\par 
By contrast, relatively less work has been published on automatic ways of making diagnosis with \ac{rcm} despite their promising results in a clinical context with specialists. In these works, several researches use either spatial features based on Gray Level Co-occurrence Matrix~\cite{Wiltgen2008}~\cite{Koller2011}, or frequency features such as wavelet-based approach for instance~\cite{Wiltgen2008}~\cite{Koller2011}~\cite{Halimi2017a}.\par 
This paper proposes a methodology devoted diagnosis of Lentigo benign and malignant form. A recent research paper reported a similar problematic while focusing on pertinence of the Wavelets for classification into Healthy and Lentigo skin pathology~\cite{Halimi2017a}. We use this work as reference and improve it by extending the classification to three classes: Healthy skin, Benign and Malignant Lentigo. We also implement two different methods to extract features by use of another hand-crafted descriptor called “Haralick", and an automatic one using \ac{cnn}.\par
Usually, classification of images (categorized as containing a pathology or not) and especially with learning-based methods is achieved considering the whole image. This makes it practically impossible to link the decision to the spatial content of the image. To overcome this drawback, we propose to build classifiers operating on local areas of an image and merge their decisions to deliver the final decision.\par
In the next section we describe the proposed methodology along with the carried out experiments. Then in Section III, we analyze and discuss the results obtained over a dedicated dataset we have built. Finally, we conclude on this work.\par